%!TEX TS-program = lualatex
%!TEX encoding = UTF-8 Unicode

\documentclass[12pt, hidelinks]{exam}

%\printanswers

\usepackage{graphicx}
	\graphicspath{{/Users/goby/Pictures/teach/163/lab/}
	{img/}} % set of paths to search for images

\usepackage{geometry}
\geometry{letterpaper, left=1.5in, bottom=1in}                   
%\geometry{landscape}                % Activate for for rotated page geometry
\usepackage[parfill]{parskip}    % Activate to begin paragraphs with an empty line rather than an indent
\usepackage{amssymb, amsmath}
\usepackage{mathtools}
	\everymath{\displaystyle}

\usepackage{fontspec}
\setmainfont[Ligatures={TeX}, BoldFont={* Bold}, ItalicFont={* Italic}, BoldItalicFont={* BoldItalic}, Numbers={Proportional, OldStyle}]{Linux Libertine O}
\setsansfont[Scale=MatchLowercase,Ligatures=TeX, Numbers={Proportional,OldStyle}]{Linux Biolinum O}
\setmonofont[Scale=MatchLowercase]{Linux Libertine Mono O}
\newfontfamily{\liningnum}[Numbers=Lining]{Linux Libertine O}
\usepackage{microtype}%

\usepackage[table]{xcolor}

\usepackage[bold-style=ISO]{unicode-math}
\setmathfont[Scale=MatchLowercase]{Tex Gyre Pagella Math}

\usepackage{tikz}

\usepackage{pifont} % Use the x mark in the final section.

\usepackage{booktabs}
\usepackage{multicol}


\usepackage{caption}
\captionsetup{format=plain, justification=raggedright, singlelinecheck=off,labelsep=period,skip=3pt} % Removes colon following figure / table number.

%\usepackage{caption}
%\captionsetup{font=small} 
%\captionsetup{singlelinecheck=false}
%\captionsetup[figure]{labelsep=period, format=plain}

\usepackage{longtable}
\usepackage{caption}
\captionsetup{format=plain, justification=raggedright, singlelinecheck=off,labelsep=period,skip=3pt} 

\usepackage{array}
\newcolumntype{L}[1]{>{\raggedright\let\newline\\\arraybackslash\hspace{0pt}}p{#1}}
\newcolumntype{C}[1]{>{\centering\let\newline\\\arraybackslash\hspace{0pt}}p{#1}}
\newcolumntype{R}[1]{>{\raggedleft\let\newline\\\arraybackslash\hspace{0pt}}p{#1}}

\usepackage{enumitem}
\setlist{leftmargin=*}
\setlist[1]{labelindent=\parindent}
\setlist[enumerate]{label=\textsc{\alph*}.}
\setlist[itemize]{label=\color{white}\textbullet}

\usepackage{hyperref}
%\usepackage{placeins} %PRovides \FloatBarrier to flush all floats before a certain point.
\usepackage{hanging}

\usepackage[sc]{titlesec}

%% Commands for Exam class
\renewcommand{\solutiontitle}{\noindent}
\unframedsolutions
\SolutionEmphasis{\bfseries}

\renewcommand{\questionshook}{%
	\setlength{\leftmargin}{-\leftskip}%
}

%Change \half command from 1/2 to .5
\renewcommand*\half{.5}

\pagestyle{headandfoot}
\firstpageheader{\textsc{bi}\,063 Evolution and Ecology}{}{\ifprintanswers\textbf{KEY}\else Name: \enspace \makebox[2.5in]{\hrulefill}\fi}
\runningheader{}{}{\footnotesize{pg. \thepage}}
\footer{}{}{}
\runningheadrule

\newcommand*\AnswerBox[2]{%
    \parbox[t][#1]{0.92\textwidth}{%
    \begin{solution}#2\end{solution}}
    \vspace{\stretch{1}}
}

\newenvironment{AnswerPage}[1]
    {\begin{minipage}[t][#1]{0.92\textwidth}%
    \begin{solution}}
    {\end{solution}\end{minipage}
    \vspace{\stretch{1}}}

\newlength{\basespace}
\setlength{\basespace}{5\baselineskip}


\newcommand\chisq{$\chi^2$}
\newcommand*\meanY{\overline{Y}\kern0.67pt}

\newcommand*\AnswerBlank[1]{%
	\ifprintanswers%
		\textbf{#1}
	\else%
		\rule{0.75in}{0.4pt}\kern0.67pt.\fi%
	}

%\newcommand*\AnswerBlank{\rule{0.75in}{0.4pt}\kern0.67pt.}
\newcommand*\xcell[1]{cell~\liningnum{#1}}

%
%\makeatletter
%\def\SetTotalwidth{\advance\linewidth by \@totalleftmargin
%\@totalleftmargin=0pt}
%\makeatother



\begin{document}

\subsection*{Resource partitioning: \textit{Bombus} bumble bees}

Resource partitioning allows more species to co-occur in a community.
Coexisting species that use one or more resources in the same way compete
with each other for the resources. Competition is reduced if species
use resources in different ways.

\textbf{Build but keep it short}

Equations for you to recall:


where,

\begin{itemize}
	\item $N_1$ is the estimated population size,
	\item $M_1$ is the number of marked individuals released into the
	population from the first capture,
	\item $n_2$ is the number of
	individuals taken in the second sample, and
	\item $M_{12}$ is the
	number of individuals in the second sample that were marked during the
	first sample.
\end{itemize}


You will analyze a variety of data to explore resource partitioning for five
species of bumble bees in the genus \textit{Bombus.} These data were collected
by Macior (1974), Pyke (1982), and Pyke et~al.~(2012) near Crested Butte, Colorado,
southwest of Denver. 

Researchers walked transects (paths) counting species of bumble bees and the 
species of flowers visited by the bees for pollen and nectar. They walked
three transects and counted a total of 13,136 individuals for 12 species of 
\textit{Bombus} (Pyke 1982).

For simplicity, you'll use data for five of the species most commonly sighted along two of the transects.


\subsubsection*{Analysis: proboscis lengths}

Download bombus\_data.xlsx\footnote{All data currently in one
spreadsheet.} and click on the “Proboscis Lengths” tab. The sheet
has proboscis lengths from 50 individuals for each of five \textit{Bombus} species. All measurements are in millimeters (mm).

Use the \texttt{average} function in Excel to calculate the mean $\left(\meanY\right)$ proboscis length for each species. Use the \texttt{stdev.s} function
to calculate the sample standard deviation $\left(s\right)$ for each species. %Use the \texttt{count} and \texttt{sqrt} functions to calculate standard 
%error of the mean. As a reminder, the equation to calculate $\mathrm{SE}_{\meanY}$ is
%
%\begin{equation*} \label{eq:stderr}
%\mathrm{SE}_{\meanY} = \dfrac{s}{\sqrt{n}}
%\end{equation*}
%
%where $s$ is the standard deviation of the sample and $n$ is the sample size. 

\newpage

\begin{questions}

\question
Fill in the blanks for the five \textit{Bombus} species.

\begin{tabular}{@{}lcc@{}}
\toprule
Species & $\meanY$ & $s$ \tabularnewline
\midrule
& &  \tabularnewline
\textit{B.~appositus} & 
\ifprintanswers \textbf{12.8} \else \rule{1in}{0.4pt} &
\ifprintanswers \textbf{0.4} \else \rule{1in}{0.4pt}  
\tabularnewline[2em]
%
\textit{B.~bifarius} &
\ifprintanswers \textbf{8.5} \else  \rule{1in}{0.4pt} &
\ifprintanswers \textbf{0.4} \else \rule{1in}{0.4pt} 
\tabularnewline[2em]
%
\textit{B.~frigidus} & 
\ifprintanswers \textbf{7.3} \else \rule{1in}{0.4pt} &
\ifprintanswers \textbf{0.3} \else \rule{1in}{0.4pt} 
\tabularnewline[2em]
%
\textit{B.~kirbiellus} &
\ifprintanswers \textbf{12.1} \else \rule{1in}{0.4pt} & 
\ifprintanswers \textbf{0.4} \else \rule{1in}{0.4pt} 
\tabularnewline[2em]
%
\textit{B.~sylvicola} & 
\ifprintanswers \textbf{8.4} \else \rule{1in}{0.4pt} & 
\ifprintanswers \textbf{0.5} \else \rule{1in}{0.4pt} 
\tabularnewline

\bottomrule
\end{tabular}

\bigskip

\question
Sketch a graph of the means with the standard deviations. Mark a 
large point above each species name at the proper height for the 
mean. Draw a thin vertical line that extends above and below the 
mean by the amount of the deviation. For example, if the mean is 
10.0~mm and the standard deviation is 0.3~mm, mark a point at 
10.0, and then draw a thin line from 9.7 (10.0\,$-$\,0.3) to 10.3 (10.0\,$+$\,0.3). \textbf{Include example on graph}


\begin{tikzpicture}
\draw[help lines, color=gray!30, step=0.25cm] (-7,-6) grid (7,6);

\end{tikzpicture}


\subsubsection*{Second capture-mark-release}\label{sec:second_mark}



\question[Checkout]
If migration occurred in a natural population being studied, how
would this influence the reliability of your estimate of population size
determined using the mark-and-recapture technique?


\end{questions}


\subsubsection*{Literature Cited}

\begin{hangparas}{\leftmargin}{1}

Macior, L.\,W. 1974. Pollination ecology of the front
range of the Colorado Rocky Mountains. Melanderia 15: 1–59.

Pyke, G.\,H. 1982. Local geographic distributions of bumblebees 
near Crested Butte, Colorado: competition and community structure.
Ecology 63: 555–573.

Pyke, G.\,H., D.\,W.\,Inouye, and J.\,D.\,Thomson. 2012. Local geographic distributions of bumble bees near Crested Butte, Colorado: competition and community structure revisited. Environmental Entomology 41: 1332–1349.

\end{hangparas}

\end{document}  